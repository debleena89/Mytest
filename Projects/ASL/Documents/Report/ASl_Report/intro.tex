\section{Introduction}
Software Verification has become a must necessary step of Software engineering.
All we need to have a code that is bug free. Now what is a bug ? when a code 
gives unwanted result we can say that there is a bug in the code for which we 
are not getting desired result. How can we detect that there is a bug in the 
code. During the Software development life cycle after development of the code
the code undergoes testing phase. That basically validates the code for some 
input set. But it is not possible to test all input set for a particular code.
Let us consider the following 
code:--

\begin{verbatim}
void main()
{
  int x, y;  
  
  assert((x-y>0) == (x>y));
     
}
\end{verbatim}

In the above code we are concluding that if the difference of any two integer
number be greater than 0 then subtrahend is greater than the minuend. While 
testing this small code we will take some integer value and check but it is 
very unlikely that we can consider x = 1935052801 and Y = -229208062, for this
condition our conclusion is not correct. Software Validation does not guarantee
bug free code. Because Exhaustive testing is not possible by this traditional 
methods. 

Now-a-days it is considered that if a code goes through exhaustive testing that
is for all possible inputs (that can be the total range of the data type) and
for each case it gives correct result (gives result according to the specification) 
then only it is said bug free code. This is the aim of Formal Verification.

Software Verification gives sounds result but is undecidable. That is if it says 
that a code is correct then there cannot be any bug in that code as it checks for 
all possible inputs. But for undecidable problems Formal verification cannot give
correct result in a finite amount of time. It can find bug up to a definite range.
Bugs that are beyond that range are not found. 

\section {Bounded Program Analysis}

An execution path consists of sequence of instructions executed during a single 
run of the program. The states observed along the execution path are called 
trace. For different inputs many such different traces can be there in the 
program. We can represent each path with formula: for particular inputs 
assignments along that trace will be satisfied. These formula are called 
symbolic simulation. The useful application of symbolic simulation can be
automatic test generation, detection of dead code and verification of asserted
 properties.

With a view to avoiding undecidability, we can impose bounds on the loops. Thus 
we can restrict the unwinding of the loop upto our desirable depth. Then we can
construct a logical formula that gives a clear picture of the relationship 
between input and output variables. Here comes static single assignment that is
a variable with time stamp that signifies a variable used once in the code
and says the value of that variable at that particular time.\\

let us consider an example of loop, calculation of factorial of a number:--

\noindent

1 void factorial ( int n )\\

2 \{\\

3 int i = 1;\\

4 int product = 1;\\

5 while ( i <= n )\\

6 \{\\

7 product = product * i;\\

8 i = i + 1;\\

9 \}\\

10 \}\\

The following table shows the sequence of instruction that comes through the
execution of the above code. It has 3 Columns, First Column says the line 
number, second Column says the type of the instruction and the third Column
shows the instruction or condition:--

\begin{center}

\begin{tabular}{|| c | c | c ||}

\hline

Line No. & Type & statement or condition \\ [0.5ex]

\hline

3 & assignment & i = 1; \\


4 & assignment & product = 1; \\


5 & branch & i $<$ n; \\

7 & assignment & product = product * i; \\


8 & assignment & i = i + 1; \\ [1ex]

\hline

\end{tabular}

\end{center}

\noindent

The above statements can be represented using single static assignment in the
following way.

For this we are considering that n = 2.

\begin{center}

\begin{tabular}{|| c | c | c ||}

\hline

Line No. & Type & statement or condition \\ [0.5ex]

\hline

3 & assignment & $ i_0 = 1; $ \\


4 & assignment & $ product_0 = 1; $ \\


5 & branch & $ i_0 $ $<$ $ n ;$ \\

7 & assignment & $product_1 = product_0 * i_0; $ \\


8 & assignment & $ i_1 $ $=$ $ i_0 + 1; $ \\

5 & branch & $ i_1 $ $<$ $ n ;$ \\

7 & assignment & $product_2 = product_1 * i_1; $ \\


8 & assignment & $ i_2 $ $=$ $ i_1 + 1; $ \\

5 & branch & $ i_2 $ $<$ $ n ;$ \\ [1ex]

\hline

\end{tabular}

\end{center}

In the above table assignments are written using single static assignment. When
i is first used in line number 3, it is denoted as $ i_0 $. From line 3 to 7 of
the code the value remain same. In line number 8 the value of i is incremented 
to 1. now this new value of i is assigned in $ i_1 $. So it is called static 
single assignment where each variable will be assigned only once.

This SSA form is now translated into a logical formula called the \textbf{path constraint}.
The formula is shown below
ssa $\Longleftrightarrow$ $ i_0 = 1; $                        $\wedge $ \\
                          $ product_0 = 1; $                  $\wedge $ \\
                          $ i_0 $ $<$ $ n ;$                  $\wedge $ \\
                          $product_1 = product_0 * i_0; $     $\wedge $ \\
                          $ i_1 $ $=$ $ i_0 + 1; $            $\wedge $ \\
                          $ i_1 $ $<$ $ n ;$                  $\wedge $ \\
                          $product_2 = product_1 * i_1; $     $\wedge $  \\
                          $ i_2 $ $=$ $ i_1 + 1; $            $\wedge $ \\ 
                          $ i_2 $ $<$ $ n ;$                  $\wedge $ \\
                          
\section {Assertion }

While we write a program we want to perform some functionality. We can consider
that a big task or function is an integration of small task. If these small 
small tasks can be handled successfully then the big task can be solved 
confidently. These small tasks or situation can be checked using Assertion.

If we could handle that situation properly in the code and we invoke assertions
for that situation it will pass that means we handled that situation properly 
otherwise the assertion will fail. In case of assertion, each run of the code is 
explored to check whether the properties hold or not.

For example , Let us consider the following assert:
 \begin{verbatim}
  int a, b;
  assume(a = 6);
  assert(b < a);
 \end{verbatim}

 The above written assertions are valid Because as the range of b is not specified
 it can be any number less than 6. But consider the following assertion :
 
 \begin{verbatim}
  int a, b;
  assume(a = 6);
  assume(b > 10);
  assert(a > b);
 \end{verbatim}

 This assertion is not satisfiable as b cannot be greater than 10 and less than 6
 at the same time. 
 

