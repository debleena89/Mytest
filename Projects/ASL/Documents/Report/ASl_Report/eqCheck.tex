%%%% Basic Packages %%%%%%%%%%%%%%%%%%%%%%%%%%%%%%%%%%%%%%%%%%%%%%%%%%%%%%%%%%%%%%%% 
\documentclass[11pt]{book}               % I'm using a double-sided book style 
\usepackage{graphicx,psfrag,amsfonts} 
\usepackage{comment}
%\usepackage[body={6.0in, 8.2in},left=1.25in,right=1.25in]{geometry} 
\usepackage[body={6.0in, 8.2in},left=1.25in,right=1.25in]{geometry} 
                                         % Geometry package for easy page margin 
                                         % setup 
\usepackage{amsmath,amssymb}             % AMS Math 
\usepackage{setspace}
%\usepackage{rotating}                    % Sideways of figures & tables 
%\usepackage[sectionbib]{natbib}          % Cross-reference package (Natural BiB) 
%\usepackage{chapterbib}                  % Put References at the end of each chapter 
                                         % Do not put 'sectionbib' option here. 
                                         % Sectionbib option in 'natbib' will do. 
\usepackage{fancyhdr}                    % Fancy Header and Footer 
%\usepackage{setspace}                    % Line spacing 
%\doublespacing 
%%%%%\usepackage{txfonts}                     % Public Times New Roman text & math font 
%%% Fancy Header %%%%%%%%%%%%%%%%%%%%%%%%%%%%%%%%%%%%%%%%%%%%%%%%%%%%%%%%%%%%%%%%%% 
% Fancy Header Style Options 
\pagestyle{fancy}                       % Sets fancy header and footer 
\fancyfoot{}                            % Delete current footer settings 
\renewcommand{\chaptermark}[1]{         % Lower Case Chapter marker style 
  \markboth{ \thechapter.\ #1}{}} % 
\renewcommand{\sectionmark}[1]{         % Lower case Section marker style 
  \markright{\thesection.\ #1}}         % 
\fancyhead[LE,RO]{\bfseries\thepage}    % Page number (boldface) in left on even 
                                        % pages and right on odd pages 
\fancyhead[RE]{\small \bfseries\leftmark}      % Chapter in the right on even pages 
\fancyhead[LO]{\small \bfseries\rightmark}     % Section in the left on odd pages 
%\fancyhead[RE]{\leftmark}      % Chapter in the right on even pages 
%\fancyhead[LO]{\rightmark}     % Section in the left on odd pages 
\renewcommand{\headrulewidth}{0.3pt}    % Width of head rule 
%%% Clear Header %%%%%%%%%%%%%%%%%%%%%%%%%%%%%%%%%%%%%%%%%%%%%%%%%%%%%%%%%%%%%%%%%% 
% Clear Header Style on the Last Empty Odd pages 
\makeatletter 
\def\cleardoublepage{\clearpage\if@twoside \ifodd\c@page\else% 
    \hbox{}% 
    \thispagestyle{empty}%              % Empty header styles 
    \newpage% 
    \if@twocolumn\hbox{}\newpage\fi\fi\fi} 
\makeatother 
\renewcommand{\thefootnote}{\fnsymbol{footnote}} 



\bibliographystyle {acm}

\begin{document}
\newtheorem {theorem} {Theorem} [chapter]
\newtheorem {lemma} {Lemma} [chapter]
\newtheorem {corollary} {Corollary} [chapter]
\newtheorem{example}{Example}[chapter]
\newtheorem{definition}{Definition}[chapter]
\newtheorem{defn}{Definition}[chapter]
\newtheorem{result}{Result}[chapter]
\newtheorem{observation}{Observation}[chapter]
\newtheorem{proposition}{Proposition}[chapter]
\newtheorem{assumption}{Assumption}[chapter]
%\newtheorem{theorem}{Theorem}[section]

\include{front_title}
%\begin{titlepage}
\begin {center}
\vspace*{0.4cm}


\centering
{\huge\sf \emph{Formal Methods for Accelerating Formal,
    Semi-formal and Dynamic Property Verification
    through Novel Specification Styles}
}
%
%{\centering {\Huge\sf Formal Methods for Accelerating Formal,
%	Semi-formal and Dynamic Property Verification
%	through Novel Specification Styles}
%\Huge \par}

\vspace {.6in}
{\scshape  Thesis submitted in partial fulfillment of the requirements}\\
{\scshape for the degree of}\\
\sf
\vspace {0.3in}
{\large Doctorate of Philosophy}\\

{in}\\
{\large Computer Science and Engineering}\\
\vspace {.2in}
\large
{by}\\
\vspace{.1in}
{\bf Ansuman Banerjee} (Roll No: 04CS9701) \\
\vspace {.1in}
{under the guidance of}\\
\vspace{.2in}
{\bf Prof. Pallab Dasgupta} \\
\vspace {.1in}
{and} \\
\vspace {.1in}
{\bf Prof. P.P.Chakrabarti} \\
\vspace{.1in}
\begin{figure}[htbp]
{\centering \resizebox*{!}{3cm}{\includegraphics{iit-25.eps}} \par}
%}}
%}}
\end{figure}

{
{\bf Department of Computer Science and Engineering}\\
{\bf Indian Institute of Technology Kharagpur}\\
{\bf August 2007}
}
\end {center}
\end{titlepage}

%\include{dedic}
\newpage
\newpage


\parskip=.14in
%\input {certi}
\parskip=.008in
%\include{pref}
\parskip=.14in
\pagenumbering{arabic}
%\chapter*{}
\vspace*{-2in}
%Abstract\markboth{Abstract}{Abstract}}
\begin{center}
\Large \bf {Abstract}
\end{center}
%\vspace{-.2in}

\noindent
In recent times, most leading
chip design companies are seriously investigating the possibility of
integrating property verification into their pre-silicon validation flows.
Property verification allows the designer to express the key
correctness requirements of a design in terms of formal properties and
verify them over a given implementation. 
Property verification is predominantly used in two forms in pre-silicon
validation, namely (a) dynamic property verification (DPV), and
(b) static Formal Property Verification (FPV).

\noindent
Existing property verification technology is poised at an interesting point. 
The benefits of property verification have been established quite
emphatically in the last decade. Researchers have analyzed several
historically significant failures and have shown that the use of property
verification could have detected the bug in the design.
Yet, the adoption of property verification techniques into the 
pre-silicon validation flow of chip design companies has been retarded 
by the following factors:
\begin{itemize}

\item {\em FPV Capacity:} Current FPV techniques do not scale
    beyond small circuit modules. 

\item {\em DPV Coverage:} The main criticism of the DPV approach is that 
	only those behaviors that are covered by simulation are examined for 
	property violation. 

\end{itemize}

\noindent
This thesis has two main motivations:

\begin{itemize}

\item We believe that the scalability and efficiency of formal and 
	semi-formal verification can be improved by adopting specification styles 
	that syntactically facilitate abstractions and pruning.

\item We believe that the assertion coverage in DPV 
	can be improved by guiding the simulation through formal 
	methods.
\end{itemize}
\noindent
The main objective of this thesis is to study the above issues
and propose methods for accelerating formal, semi-formal and dynamic 
property verification.
In particular, we have the following contributions:

\begin{itemize}
\item {\em Accelerating Property Coverage in DPV:} We have defined an 
	automated methodology that can analyze formal properties and produce 
	tests that trigger them. This has been integrated within a 
	constrained random test architecture to accelerate coverage of 
	corner case assertions during DPV.

\item {\em Context-sensitive specifications:} A significant fraction
    of the correctness requirements in standard protocol descriptions 
	are {\em context-sensitive}, i.e. they apply only when the 
	protocol is in specific contexts. We have formalized a modeling 
	style for expressing such properties and proposed algorithms 
	for formal and semi-formal verification 
	and consistency analysis for such properties.

\item {\em Annotating Temporal Operators for Modular FPV:}
	To ameliorate the state explosion
    problem, there has been a paradigm shift from a system level FPV
    perspective to a modular one. To capture module-level specifications, 
	we have proposed an extension of Linear Temporal Logic by annotating 
	temporal operators with input constraints. In addition, 
	we have proposed a symbolic BDD-based algorithm for verifying such 
	properties. 

\item {\em An integrated DPV platform for UML Statechart validation:} 
	We have developed a complete DPV platform for verifying temporal 
	requirements over UML Statecharts. The DPV framework allows the user 
	to specify correctness requirements in a rich assertion specification 
	language and verifies them during simulation. 

\end{itemize}
\noindent
We believe that the formal methods presented in
this thesis will lead to wider adoption of property verification techniques 
in the design validation flow.

\newpage

\tableofcontents
\newpage
\listoffigures
\newpage
%\listoftables
\newpage


%\documentclass{article}


%\begin{document}

%\title{A short survey on Equivalence Checking methods for digital circuits}

%\maketitle

\chapter{Introduction}
\par The process of equivalence checking is an important step in Electronic Design Automation (EDA), commonly used during the developments
of system design, to formally check whether two circuits exhibit exactly the 
same behavior. We begin by formally defining the notion of equivalence below.

\begin{defn}
Two circuits are said to be equivalent if their outputs match for every input sequence. 
$\blacksquare$
\end{defn}
\noindent
We explain the equivalence checking problem on the following example.

\begin{example}
\begin{figure}[!h]
\centering
\includegraphics[height=3cm, width=8cm]{equivalence.pdf}
\caption{An example of two equivalent circuits} \label{fig1}
\end{figure}

Figure \ref{fig1} shows two different circuits. Figure-\ref{fig1}(a) shows a simple and-or circuit and Figure-\ref{fig1}(b) shows a circuit
represented by NAND gates. These two designs are equivalent since they realize the same Boolean function \\ y = (ab + cd ).
$\blacksquare$
\end{example}

% \par Now the product computation is defined as the set states of the product is the cross product of the states of the individual machines. 
%That means, the global state is composed by taking one state from one machine and another state from another
%machine.Then the concatenation of the state of the two machines $M_1$ and $M_2$ is called global state. 
%The output function is defined as if in a state if the output match corresponding to all inputs then the output 
%of the product machine is 1 and the output is 0 otherwise. 

%\par If we see the output is 0 the two machines are not equivalent. That's not really the case. The case is whether
%a global state when the output do not match is reachable or not.We can take every pair of states one from $M_1$ and $M_2$ 
%forms a global state and obviously in all states the output will not match. The question is, starting from the initial state
%can we reach the global state where there is an output mismatch. Then they are not equivalent. 

%\begin{figure}[!h]
%\centering
%\includegraphics[height=4cm]{stateMachine.pdf}
%\caption{Finite State Machine} \label{fig0}
%\end{figure}

%\par  Figure-1 shows a finite state machine M(X,Y,$S_0$ , $S_1$, $\delta$, $\lambda$), where X is an input, Y is output, S is current 
%state, $S_0$ is initial state, $\delta$ is a next state function which is defined as $\delta \colon X \times S \longrightarrow S $
%and $\lambda$ is an output function which is defined as $\lambda \colon X \times S \longrightarrow Y $

\noindent Equivalence checking is used when the design goes through any  transformation in 
the VLSI design life cycle (e.g. high level synthesis, logic synthesis, technology mapping etc.) to
check whether the functionality of the transformed design is equivalent to the original one. Depending on the nature
of the circuits, combinational or sequential equivalence checking is employed. We explain each of these techniques in 
the following discussion.

\input{combEq.tex}
\input{seqEq.tex}
\input{toolEq.tex}
\input{abcEq.tex}
\newpage
\input{jasperEq.tex}

\bibliographystyle{IEEEtran}
\bibliography{ref}{}

\end{document}
 
