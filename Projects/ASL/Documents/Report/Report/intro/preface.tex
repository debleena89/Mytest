
\chapter{Introduction}

Formal property verification (FPV) technology is currently poised at an
interesting position. Most chip design companies admit today that the new 
technology has several desirable features that may significantly benefit 
pre-silicon validation. On the other hand, very few companies have actually 
integrated formal or semi-formal property verification into their mainstream 
validation flow. Understanding the role of FPV in the pre-silicon validation 
flow of a typical digital design is nontrivial, since there are several open
issues for which adequate tools are not available. There are broadly three
different types of views on this matter, namely:
\begin{enumerate}

\item {\em The FPV user's point of view.} Studying this view shows us the
	limitations of existing tools and technology.

\item {\em The EDA point of view.} Studying this view shows the opportunities
	in developing new tools that can cover the gaps in existing
	technology.

\item {\em The design manager's point of view.} Studying this view shows
	the difficulties in introducing FPV into the simulation based 
	validation flow that has been practised for years.

\end{enumerate}
This book makes a modest attempt to address all three points of view, and 
proposes a roadmap for the use of FPV in pre-silicon validation. 

There is also a fourth view on this subject, ignoring which would not be 
fair -- that of the theoretician (read as university professor). However, 
to shield the gentle reader from the aggressive use of symbolic notation 
that is the trademark of this fourth category, we quarantine the admittedly 
terse theoretical foundations of the subject to special sections. These 
sections should be religiously avoided by all who do not intend to do a PhD on 
the subject.

\section{The Property Verification Frameworks}
{\em What is formal property verification?} The ambiguity admitted by the
English language permits two different ways of interpreting {\em formal
property verification}, namely:
\begin{itemize}

\item {\em A formal approach to property verification}, or
\item {\em A verification approach for formal properties.}

\end{itemize}




sections, so that the  


Since we are concerned with {\em formal} methods, there is a notable fourth

The focus of this book
is on identifying these issues from the FPV user's point of view, on 
investigating existing and new solutions to these problems from a Electronic
Design Automation (EDA) point of view, and on attempting to present a
roadmap for integrating FPV within a simulation based validation flow from
a design manager's point of view. The university Professor's point of view 
has been relegated to special sections -- the quarantine being necessary
in view of the generous use of symbolic notations which are quite 
intimidating to the gentle reader.





key features of the new technology


holds promise, 

The use of property verification technology in the validation of VLSI circuits
evolved in 
 
Property verification technology has reached a stage where many of the leading
chip design companies are actively considering the possibility of integrating
this technology into their mainstream design validation flow, instead of 

 Pre-silicon validation accounts for more than 70%
