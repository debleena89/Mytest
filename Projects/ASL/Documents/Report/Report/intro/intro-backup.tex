
\chapter{Introduction}

{\em What is formal property verification?} There are two ways in which English
allows us to interpret the term {\em formal property verification}, namely:
\begin{itemize}

\item {\em Verification of formal properties}, or
\item {\em Formal methods for property verification}

\end{itemize}
Formal property verification relates to both of these interpretations. On the
other hand, design specifications in English are not always so generous -- 
it is very rare to have a specification that can be interpreted in more than
one way and both interpretations are acceptable. In an era where bugs are more
costly than transistors, the industry is beginning to realize the value of 
using formal specifications.

Property verification is taking shape in two different forms within the design
validation framework. These are {\em Formal Property Verification (FPV)} and
{\em Assertion-based Verification (ABV)}. In both forms, formal properties 
specify the correctness requirements of the design, and the goal is to check 
whether a given implementation satisfies the properties. FPV techniques 
formally verify whether {\em all} possible behaviors of the design satisfy the 
given properties. ABV is a simulation-based approach, where the properties are 
checked over a simulation run -- the verification is thereby confined to only 
those behaviors that are encountered during the simulation. FPV is often
refered to as {\em static property verification}, while ABV is synonymous
with {\em dynamic property verification}. 

The main tasks for a practitioner of property verification are as follows:
\begin{enumerate}

\item Development of the formal property specification. The main challenge
	here is to express key features of the design intent in terms of
	formal properties.

\item Verifying the consistency and completeness of the specification. This
	is a necessary step, because the first task is a non-trivial one and 
	subject to errors and oversights.

\item Verifying the implementation against the formal property specification.
	In order to perform this task effectively, a validation engineer must
	be aware of the limitations of the validation tool and must know
	the best way to use the tool under various types of constraints.

\end{enumerate}
All the above tasks are replete with open issues -- the focus of this book is 
to consider some of these issues and attempt to forecast the roadmap for FPV 
and ABV within the existing design validation flows of chip design companies.
This chapter will summarize some of the major challenges. Let us use the
following case as a running example for our discussion.

\begin{example} \label{ex1}
{\em
Let us consider the specification of a 2-way priority arbiter having the 
following interface:
\begin{tabbing}
aa \= aa \= aa \= \kill
\> mem-arbiter( input $r_1$, $r_2$, $clk$, non-input $g_1$, $g_2$ )
\end{tabbing}
$r_1$ and $r_2$ are the request lines, $g_1$ and $g_2$ are the corresponding
grant lines, and $clk$ is the clock on which the arbiter samples its inputs
and performs the arbitration. We assume that the arbitration decision for the
inputs at one cycle is reflected by the status of the grant lines in the
next cycle. Let us suppose that the specification of the arbiter contains 
the following requirements:
\begin{enumerate}

\item Request line $r_1$ has higher priority than request line $r_2$. Whenever
	$r_1$ goes high, the grant line $g_1$ must be asserted for the
	next two cycles.

\item When none of the request lines are high, the arbiter parks the grant
	on $g_2$ in the next cycle.

\item The grant lines, $g_1$ and $g_2$, are mutually exclusive.

\end{enumerate}
It is difficult to locate a book on formal verification that does not have
an arbiter example - we hereby follow the tradition!
} $\Box$
\end{example}

\section{Writing Our First Formal Specification}
The first task in all forms of property verification is the development of
the formal specification. This is a non-trivial task and typically done by
a few specialized validation engineers today. The availability of several
different languages for property specification and relatively few tools that
support these languages adds another dimension to the complexity of this task.

In recent times there has been a concerted effort at various forums to 
unify and standardize the syntax and semantics of property specification 
languages. In spite of this effort, there are still several competing
languages, including Sugar/PSL of IBM/Acellera, System Verilog Assertions (SVA)
of Accelera, Forspec of Intel and Open Vera Assertions (OVA) of Synopsys.
Fortunately, the industry appears to be moving towards unifying these
languages and adopting a mutually acceptable form.

{\em What is in these languages that was not there in earlier languages for
specifying constraints?} The main feature of these languages, which makes 
them useful in practice for design verification, is the ability to describe
sequences of events over time. 
The property specification languages derive their formalism from few specific
types of logics called {\em temporal logics}. Temporal logics are extensions
of propositional logic, where in addition to the familiar Boolean operators
(AND, OR, and NOT) we have {\em temporal operators} that allow us to specify
constraints on the truth of the propositions over time. In other words,
temporal logics allow us to specify properties that describe the behavior of
a circuit over time, across cycle boundaries.

To get a first-cut workable understanding of temporal logics, let us consider
some of the basic temporal operators and their meaning. The operators are
interpreted over a state machine -- the purpose of verification is to interpret
the properties over the state machine representation of the design
implementation. The details on temporal operators and property specification
languages are given in Chapter~\ref{chap2}. We use illustrative examples
here to demonstrate their use. The basic set of temporal operators in
{\em Linear Temporal Logic} (LTL) are:
\begin{description}

\item[$X$: {\em The next-time operator}] The property, $X\varphi$, is true
	at a state if $\varphi$ 
	is true in the next cycle, where $\varphi$ may be another temporal 
	property or a Boolean property over the state bits. $X\varphi$ is 
	sometimes read as {\em "next f"}, and the operator $X$ is called 
	the {\em next-time operator}.

\item[$F$: {\em The future operator}] The property, $F\varphi$, is true at a
	state if $\varphi$ is true {\em sometime} in the future.

\item[$G$: {\em The global operator}] The property, $G\varphi$, is true at a
	state if $\varphi$ is true {\em always} in future.

\item[$U$: {\em The until operator}] The property, $\varphi\ U\ \psi$ is
	true at a state if $\psi$ is true at some future state, $t$, and 
	$\varphi$ is true at all states leading upto $t$.

\end{description}
The operators $X$ and $U$ are the only fundamental temporal operators -- $F$ 
and $G$ can be derived from combinations of $U$ and Boolean operators.

\begin{example} \label{ex2}
{\em 
Let us attempt to encode the properties described in Example~\ref{ex1} using
LTL. Let us recall the properties:
\begin{enumerate}

\item Request line $r_1$ has higher priority than request line $r_2$. Whenever
        $r_1$ goes high, the grant line $g_1$ must be asserted for the
        next two cycles.

\item When none of the request lines are high, the arbiter parks the grant
        on $g_2$ in the next cycle.

\item The grant lines, $g_1$ and $g_2$, are mutually exclusive.

\end{enumerate}
The first property may be written as:
\[ G[\ r_1 \Rightarrow\  Xg_1\ \land\  XXg_1\ ] \]
The subformula $r_1 \Rightarrow\  Xg_1\ \land\  XXg_1$ says: {\em If $r_1$
is high in some state, then $g_1$ must be true in the next cycle and $g_1$
must be true in the next to next cycle}. The $G$ operator says that the
above subformula must hold on all states. This does not mean that $r_1$
has to be true in all states -- those states where $r_1$ is low satisfy the
implication {\em vacuously}.

The second property can be written as:
\[ G[\ \neg r_1\ \land\ \neg r_2\ \Rightarrow\  Xg_2\ ] \]
The meaning of this property is exactly as before -- {\em in every state,
if both $r_1$ and $r_2$ are low, then $g_2$ must be high in the next cycle}.

The third property can be written as:
\[ G[\ \neg g_1\ \lor\ \neg g_2\ ] \]
This property says: {\em always at least one among $g_1$ and $g_2$ must be
low}, which expresses the mutual exclusion requirement.
} $\Box$
\end{example}

Having written our first formal specification, we are now in a position to
look at some of the fundamental questions that cross the minds of a 
practitioner of FPV. The answers to these questions form the major contents
of this book.

\section{Is my specification correct?}
The first major challenge faced by every validation engineer who uses FPV is
to ascertain whether the specification itself is correct. Functional
correctness is very difficult to check since we do not have any formal
reference against which we may perform this verification. However it is
possible to check whether the specification is inherently consistent, or
whether there are contradictions within the specification itself. This is
a task of considerable importance, but EDA support is not yet adequate. 

\begin{example} \label{ex3}
{\em 
Let us examine our first specification. {\em Is it consistent?} Let us examine
the properties again.
\begin{tabbing}
aaaa \= aaaa \= \kill
\> $G[\ r_1 \Rightarrow\  Xg_1\ \land\  XXg_1\ ]$ \\
\> $G[\ \neg r_1\ \land\ \neg r_2\ \Rightarrow\  Xg_2\ ]$ \\
\> $G[\ \neg g_1\ \lor\ \neg g_2\ ]$
\end{tabbing}
Let us consider the scenario, where $r_1$ is high at time $t$ and low at time
$t+1$, and $r_2$ is low at both time steps. The first property requires 
$g_1$ to be high at time $t+2$, where as the second property requires
$g_2$ to be high at time $t+2$ because both $r_1$ and $r_2$ are low at
$t+1$. The third property prevents both $g_1$ and $g_2$ to be asserted at
time $t+2$, leading us to a contradiction. Hence we have an inconsistency
in our first specification!
} $\Box$
\end{example}

Detecting inconsistencies in specifications is a non-trivial task. It can
be modeled as a game between the module and the environment, where the 
module attempts to satisfy the specification by setting appropriate values
to its outputs while the environment attempts to refute the specification
by setting values to the input signals. This game alternates between the
module and its environment over time -- the specification is inconsistent
if the environment always has a winning strategy. Chapter~\ref{consistency}
presents recent research on consistency checking methods for formal property
specifications.

Before we proceed further, let us remove the inconsistency from our first
specification. The intent of the second property was to specify that $g_2$
is the default grant. Another way to specify the same intent is:
\[ G[\ \neg g_1\ \Rightarrow\  g_2\ ] \]
This says that $g_2$ gets the grant whenever we are not committed to give the
grant to $g_1$. Henceforth we will use this property in place of the second
property in our specification.

Let us see how this eliminates the problem. If $r_1$ is high at time $t$ and
low at time $t+1$, then the first property requires $g_1$ to be true at time 
$t+1$. If $g_1$ is true, then the above property is vacuously satisfied
without requiring $g_2$ to be true, and hence there is no conflict.

Inconsistencies are very common in design specifications. Writing the
specifications in formal languages and performing consistency checks can
locate some of the most complex inconsistencies in the specifications
very early in the design flow, thereby saving a significant quantum of
validation (and possible re-designing) effort.

\section{Have I written enough properties?}
The popular selling point for FPV is that it is {\em exhaustive} in nature.
Since this guarantee requires the FPV tool to check all possible behaviors
of the implementation, it is often misinterpreted as 100\% coverage. In
reality FPV only guarantees that the given properties are verified over all
possible behaviors of the implementation -- it has no way of verifying
any feature that has not been expressed in terms of properties.

One of the main challenges of a validation engineer is to make sure that
the formal property specification covers all the correctness requirements
of the design. To check the extent of this coverage, we need to compare the 
set of formal properties with a reference which formally expresses the
complete set of correctness requirements. This is an instance of the
chicken-and-egg problem, since the reference itself could be used as the
formal specification.

It is for this reason that FPV coverage metrics are typically {\em structural}
in nature. In other words, since the property suite represents the first 
formal functional specification of the design, we do not have any functional
reference to compare it with, and therefore we resort to structural coverage.
The objective of these structural metrics is to expose gross gaps in the
specification. Typically a low value of these metrics indicate that more
properties need to be added, but a high value does not necessarily mean that
we have high functional coverage. This follows from the fact that low
structural coverage almost always means low functional coverage, but the
reverse is not always the case.

Most of the existing FPV coverage metrics use a given implementation as the
reference for coverage analysis. Chapter~\ref{completeness} outlines some
of these metrics. We also present a new style of FPV coverage analysis which
is based on the following criticism of existing FPV coverage metrics.
\begin{enumerate}

\item If the reference is an implementation, then the coverage of a property
	suite will change with the implementation. An incomplete implementation
	may give a false sense of high coverage.

\item We believe that the future of FPV lies in a validation flow, where the
	properties will be written with the specification document at a time
	when implementation is yet to begin. At that stage we need to analyze
	the completeness of the specification in the absence of any reference
	implementation.

\item Existing coverage metrics have the same evaluation complexity as
	model checking. Therefore coverage analysis runs into the same
	capacity issues as model checking.

\end{enumerate}
Our style of coverage analysis compares the property suite against a fault
model. Intuitively, if there is any input or non-input signal of the
design-under-test which is a {\em don't care} with respect to every property
in the suite, then there is a coverage gap since we did not specify any
behavior involving that signal. The details of this new approach is 
presented in Chapter~\ref{completeness}. 

\begin{example} \label{ex4}
{\em
Let us again consider our first specification after the modifications in the
last section.
\begin{tabbing}
aaaa \= aaaa \= \kill
\> $G[\ r_1 \Rightarrow\  Xg_1\ \land\  XXg_1\ ]$ \\
\> $G[\ \neg g_1\ \Rightarrow\  g_2\ ]$\\
\> $G[\ \neg g_1\ \lor\ \neg g_2\ ]$
\end{tabbing}
Does this specification cover any behavior where $g_1$ is required to be high?
The answer is yes, the witness being the first property. But does it ever
enforce $g_2$ to be high? The answer is no!

This has a serious implication. Consider an implementation that never asserts
$g_2$, and always asserts $g_1$ regardless of the inputs. None of our
properties will be refuted by this implementation and we will be led to
believe that the implementation is correct. Our analysis will point out
that we need to add properties that specify those behaviors where $g_2$
is forced to be high.

Let us add the following property into our specification:
\[ G[\ \neg r_1\ \land\  X\neg r_1\ \Rightarrow\ XX\neg g_1\ ] \]
Adding this property eliminates the problem. It guarantees that $g_1$ is never
asserted except in those cases covered by the first property. The second
property forces $g_2$ to be high by default.

Let us now look at the input signals. Do we need to read $r_1$ at all? The
answer is yes. Suppose both $g_1$ and $r_2$ are low at time $t$. If $r_1$
is high, the arbiter must assert $g_1$ in the next cycle (by the first
property). On the other hand, if $r_1$ is low, then the aribiter must
not assert $g_1$ in the next cycle (by the new property). Therefore it cannot
satisfy the specification without reading $r_1$.

Can we satisfy the specification without reading $r_2$? The answer is yes!
The specification is free from $r_2$. This is another form of gap which 
points out the necessity to add properties that cover those cases where
the value of $r_2$ must be considered for setting the correct outputs.
Without complicating matters any further, let us accept the fact that in our
arbiter specification, $r_2$ is indeed redundant!
} $\Box$
\end{example}

We are now ready with a consistent and (hopefully) complete specification
of our arbiter. The next step is to use this specification to verify the
design implementation.

\section{How can I check the properties?}
Having written our first formal specification, we are now faced with the
option of using one of the two broad methodologies for property verification,
namely {\em Assertion Based Verification} (ABV) and {\em Formal Property
Verification} (FPV). Let us study the ABV approach first.

\subsection{The ABV approach}
In the ABV approach, we must first write a test bench to drive inputs into
our implementation. The complexity of this task grows rapidly with the
complexity of the design. This is because the environment of a module is
typically constrained by the behavior of the other modules in the design
and by the protocol used for their communication. For example, to verify
an endpoint device in a PCI Express architecture, the test bench must model
the rest of the architecture consisting of other endpoints, the switches and
the root complex. Even after this is done, it is not practically feasible
to write directed tests to sensitize all possible behaviors of this model.

The ABV setup is shown in Fig~\ref{fig1.2}. We simulate the implementation
with the test bench. The assertion checker reads the signals in the interface
and monitors the status of the properties. If any of the properties fail
during the simulation, the checker reports it immediately. The failure points
help the validation engineer to isolate the source of the bug.

There are two key features of ABV which explains the remarkable growth
in the penetration of this technology. Firstly, it is built over the
traditional simulation framework and requires nominal additional effort
from the validation engineer. Secondly, it does not have any major capacity
concerns, since the verification is done over the simulation run.

The main criticism of the ABV approach is that only those behaviors that
are covered by simulation are examined for property violation. The following
example demonstrates this problem on our arbiter.

\begin{example}
{\em 
The implementation of our arbiter is shown in Fig~\ref{fig1.1}. The boxes
represent flip-flops. The gates have their usual meanings. The same clock
(not shown) drives both flops. The verification problem is to determine
whether this implementation satisfies our formal specification. The
Verilog for the implementation is also shown.

Fig~\ref{fig1.3} shows a sample test bench for our arbiter. The test
bench has two processes - one generates a pattern of requests for $r_1$,
the other generates a pattern of requests for $r_2$. Recent test
bench modeling languages such as Vera and System Verilog (task blocks)
allow us to write such concurrent test benches.

The implementation of our arbiter has a bug. Consider the case when both
requests are low for two consecutive cycles. The specification demands that
the grant be parked on $g_2$ (by default), but in our implementation both
grants will be low. Though the specification guards against the error, the
bug will escape detection because our test bench never creates the relevant
cases.
} $\Box$
\end{example}

The example brings out one of the major challenges in property verification.
In view of the volume of directed tests that needs to be written in order to
achieve a meaningful level of functional coverage, the industry is moving
towards coverage driven randomized test generation. This helps in reaching
a high level of coverage in short time, but the difficult corner case
behaviors are typically left out. Formal properties target these
corner case behaviors, but ABV is not effective unless we can force the
test bench to create the relevant scenarios.

The task of manually deriving the tests that cover the scenarios that are
relevant to a formal property is a very complex one, even for the seasoned
practitioner of ABV. We therefore require automated formal methods that
can analyze a property and spin out the tests that trigger those properties.
Chapter~\ref{testgeneration} outlines some recent research on this topic.

\subsection{The FPV approach}
The FPV setup is shown in Fig~\ref{fig1.4}. At the heart of this approach we
have a {\em model checking} tool. A model checking algorithm has two main
inputs -- a formal property and a finite state machine representing the
implementation. The role of the algorithm is to search all possible paths
of the state machine for a path which refutes one or more properties. If one
exists, then the path trace is reported as the counter-example. Otherwise
the model checker asserts that the property holds on the implementation.

\begin{example}
{\em 
Let us again consider the implementation of our arbiter shown in 
Fig~\ref{fig1.1}. The state machine for the arbiter is shown in
Fig~\ref{fig1.5}. It may be noted that the input bits are also part of the
state of the arbiter. The intrinsic next state of the arbiter is a function
of the present state and present inputs, but since the inputs arriving in the
next cycle are not known a priori, the state machine is non-deterministic
in nature.

The objective of model checking is to search for a path in this state machine
that refutes one or more properties from our specification. Indeed one such
path is the path .... which refutes the property:
\[ G[\ \neg g_1\ \Rightarrow\  g_2\ ] \]
The model checker will find the refuting path. Since the path also contains
the input sequence for which the refutation occurs, it produces a complete
counter-example trace with the appropriate inputs that trigger the incorrect 
behavior of the module.
} $\Box$
\end{example}

We outline some of the popular model checking approaches in 
Chapter~\ref{chap2}. The interested reader may refer to \cite{clarke:00}
for details on model checking methods.

The main limitation of model checking technology is in capacity. There is a
popular misconception in this context -- validation engineers not conversant
with the model checking algorithms often tend to believe that the core model
checking algorithms suffer from capacity bottlenecks. This is not the case.
For example, the complexities of both CTL and LTL model checking are linear
in the size of the state space. CTL model checking is also linear in the size
of the property. LTL model checking algorithms are exponential in the size
of the properties, but there is not much of a capacity issue here, since 
typical properties written by validation engineers are small in size.

The main capacity bottleneck in model checking tools is in representing the
state machine that needs to be extracted from the RTL. The number of states
in the machine typically grows exponentially with the number of concurrent
components. For designs of moderate size this leads to state explosion.

Most of the recent research in formal property verification has been on
developing engineering solutions to the state explosion problem. Some of the
notable approaches are:
\begin{enumerate}

\item {\em BDD-based approaches.} Several model checking tools use {\em binary
	decision diagrams} to represent the state transition relation
	succinctly. Typically the size of BDDs tend to grow alarmingly
	after about 200 variables -- hence BDD based approaches face capacity
	problems on state machines having more than 100 state bits.

\item {\em Bounded model checking.} Often the validation engineer expects
	a property to pass or fail within some definite number of cycles.
	BMC based tools accept a bound on the expected length of a 
	counter-example (if one exists) and unfold the property and the
	state machine upto the given bound. The clauses generated by this
	unfolding are fed into a Boolean SAT solver and the result indicates
	the truth of the property up to the given bound. Since recent SAT
	solvers are able to handle millions of clauses, BMC can handle
	larger designs -- but the limitation is that we need to know the
	bound on the length of counter-examples. The bound can be iteratively
	increased, but the complexity of the SAT instances grow with the
	increase in the bound.

\item {\em Abstractions and Approximate Model Checking.} Often the {\em cone of
	influence} of a property covers only a fraction of the design. Some
	of the components of the design may not affect the truth of a property.
	In such cases it is possible to abstract out the relevant components
	and check the property on the reduced state machine. We may also
	abstract out the components that directly relate to the property
	and attempt to prove that the property holds in the reduced machine
	under the assumption that the inputs received from the other 
	components can be arbitrary. If the property holds under this 
	assumption, then obviously it holds on the given design. The reverse
	is not true, since the inputs under which the property fails may not
	be asserted by the other components. Hence such methods are referred
	to as {\em approximate model checking} methods.

\item {\em Assume-guarantee Verification.} 

 In order to find
such a path, model checking algorithms typically use a three step approach.
In the first step, we create the negation of the properties in the 
specification.

\section{Can I check all properties?}

\section{What have I verified?}


\section{Formal Property Verification}
Follow example.

\section{Assertion Based Verification}
Follow example.

\section{Major Challenges}

\section{The Roadmap}

\section{Book Structure}
