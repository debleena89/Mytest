\noindent
Assertions in Action-LTL are written over the model attributes, specifically, 
the data members and events.
Hence, to determine satisfaction/refutation of an assertion at a simulation 
step due to the model behavior, it is necessary to be able to sample values 
of the model attributes. 
Below, we briefly explain the issues faced by us in 
developing the dynamic ABV platform over Rhaspody and subsequently, the 
approaches adopted by us.
 
\begin{enumerate}
\item The assertion monitor 
must be able to sense all the events dispatched in the model. Whenever an 
event is dispatched in the model, the verifier should be informed about the 
event, its parameters, if any, and its target. 

\item With inception of every event, the assertion monitor 
must check if the model configuration satisfies/refutes the desired 
specification.
The assertion monitor must remember the context of verification; it must
remember what has to be checked after dispatch of each event to certify 
correctness of the model at that point of time. 

\item An interface is necessary through which model data attributes can 
be sampled by the assertion monitor. 

%\item The behavior of the assertion monitor must be robust enough to cope up
%with scheduling semantics of the Rhapsody simulation engine. This means that
%over all possible valid scheduling scenarios, the result of verification 
%must be same. More specifically, an error, if any, must be caught in each 
%and every valid schedule of Rhapsody simulation. 

\item The UML model must be minimally perturbed in course of 
building the assertion monitors. Last but not the least, the assertion 
monitors must be seamlessly integrable into an already existing design 
environment over Rhapsody.
\end{enumerate}

\subsection{Our approach}
\noindent
We now briefly discuss about the strategies adopted by us in addressing the 
above issues over Rhapsody. 

\begin{itemize}

\item We have overloaded a Rhapsody macro, named RiCGEN, to make the 
assertion monitor aware of the event transaction in the model. Whenever 
an event 
is dispatched in the original model, a copy of the event is instantaneously 
delivered to the assertion monitor.

\item The assertion monitor has been implemented as a Rhapsody object with an 
embedded C routine. At every simulation step, on notification of an 
event dispatch, the assertion monitor samples the data attribute values 
and checks for satisfaction/refutations. The C routine, based on the 
one step unfolding of the temporal properties as discussed in the 
previous section, has been developed and a 
control logic has been designed that calls that routine appropriately. 

\item To be able to read the values of the public data attributes of each 
object at each simulation step, the assertion monitor object is integrated 
at the appropriate class hierarchy that reveals the basic building components 
of the assertion monitor and the relationships that the individual components 
share among themselves and with the model elements. These relationships explain 
how the monitor can access the model data attributes during simulation. 

\item For integrating the assertion monitor with the Rhapsody model under 
test, one needs to have access permission inside the class hierarchy 
to be able to insert the assertion monitor object. No modification of the 
original model is needed for this purpose. 
\end{itemize}

\noindent
We first discuss the RiCGEN overloading operation and its contribution in our
work. 

\subsubsection{RiCGEN Overloading} 
RiCGEN is a macro that is the core of the event transfer mechanism of Rhapsody.
In the code base, the sender element (actor, class etc.) of an event calls this
 macro with the name of the target element and the event, with parameters if
applicable. By overloading this macro, we achieve that whenever an event is 
thrown targeting any model element, a copy of the same event with same 
parameters, if any, would go to the {\em assertion monitor}.
\footnote{The actual monitor component which receives this copy of any event 
is mentioned in the architecture}. The overloaded RiCGEN routine 
is given below.

\vspace{0.1in} 

{\tt
\noindent \#define RiCGEN(INSTANCE,EVENT)\hfill  \\
\{				\hfill  \\
\indent	if ((INSTANCE) != NULL) \hfill  \\
\indent	\{ 			\hfill  \\						\indent \indent	RiCReactive * reactive = \&((INSTANCE)->ric\_reactive);	\hfill  \\
\indent \indent	RiCEvent * event = \&(RiC\_Create\_\#\#EVENT->ric\_event);\hfill  \\
\indent \indent	RiCEvent * eventCopy = \&(RiC\_Create\_\#\#EVENT->ric\_event);\hfill  \\
\indent \indent RiCReactive\_gen(reactive, event, RiCFALSE);\hfill  \\
\indent \indent	RiCReactive\_gen( \&((\&obsrvEventReceiver)->ric\_reactive), \hfill \\ 
\indent \indent \indent \indent \indent \indent \hfill eventCopy, RiCFALSE);	 \\
\indent	\}	\hfill \\
\}\\
}

\noindent
The element {\tt obsrvEventReceiver} is a component object of the assertion 
monitor. Its 
significance and role in the object hierarchy is discussed below.
Two important objectives have been achieved in this 
implementation of the overloading:

\begin{itemize}
\item Whenever this overloaded macro is called (by some model
element to throw an event to some other), a copy of the same event is
thrown to the object {\tt obsrvEventReceiver}. This makes the assertion 
monitor aware of the inception of the corresponding event.

\item Once this macro is called, two instances of the same event
are pushed to the event queue consecutively. This is important, since, as 
mentioned previously, Rhapsody assumes a RTC semantics and at every step, 
a single event is dispatched. Whenever 
a dispatched event causes the model to move from one stable configuration to 
another, the next event notifies the monitor which then performs the 
desired task. 
\end{itemize}

\subsubsection{Assertion Monitor Architecture}\label{checkArch}
\noindent
The assertion monitor is a component that monitors the simulation 
of the UML model for any property violation. 
At the heart of the monitor lies an object named {\tt obsrvEventReceiver}. 
This object performs all the verification tasks including the 
reception of the copies of events dispatched in the model via overloaded 
RiCGEN macro. The dynamic behavior of this object is shown in the Statechart 
of Figure~\ref{fig8.5}.

\begin{figure}[htbp]
\begin{center}
\psfrag{s1}{{\tt Init}}
\psfrag{s2}{{\tt WaitForEvent}}
\psfrag{s3}{{\tt SampleVal}}
\psfrag{s4}{{\tt Check}}
\psfrag{event}{{\tt event}}
\includegraphics[scale=0.95]{../journal07/oER.pstex}
\end{center}
\caption{The Statechart for {\tt obsrvEventReceiver}} \label{fig8.5}
\end{figure}

\noindent
Tasks of the four states of the object {\tt obsrvEventReceiver} are shown 
shown in Table~\ref{tab:oER}. The tansition, labeled with {\tt event}, from 
state {\tt WaitForEvent} to state {\tt SampleVal} of Figure~\ref{fig8.5} 
signifies that control would jump from {\tt WaitForEvent} to {\tt SampleVal}
when a copy of any event dispatched in the model will reach the object 
{\tt obsrvEventReceiver} via RiCGEN. 

\begin{table}[!hbtp]
\caption{The tasks of different states of {\tt obsrvEventReceiver}}
\label{tab:oER}
\begin{center}

\begin{tabular}{|c|p{4.5in}|}
\hline
State Name & Responsibility \\
\hline \hline
{\tt Init} & Initialization of \\
	& data-structures for internal housekeeping\\
\hline
{\tt WaitForEvent} & Control waits for any event dispatch in the model\\
\hline
{\tt SampleVal} & Immediately after dispatch of any event in the model, control samples values of model attributes necessary for verification\\
\hline
{\tt Check} & Control checks for any specification violation using the values 
	sampled in {\tt SampleVal} state. The salient steps are\\
	& 1. Retrieves the context of verification for the current step\\
	& 2. Reads values of the attributes relevant to the present step context
		from the set of attributes sampled in the previous state\\
	& 3. Substitutes the relevant values for the attributes to check for 
		any violation. It reports, if any violation has been detected.\\
	& 4. Otherwise, it prepares and saves the context of verification for 
		the next step.\\
\hline
\end{tabular}
\end{center}
\end{table}

