\chapter{Conclusions and Future Work} \label{chap9}
The focus of this thesis was to study the role of formal specifications in 
the different facets of property verification. In the process of this 
exercise, we came up with new techniques for specification analysis and 
an arsenal of novel specification styles that can facilitate larger 
penetration of property verification into the validation flows of companies. 
In the light of the challenges identified in Chapter~\ref{chap1}, we believe 
that this thesis fulfills the following objectives:

\begin{itemize}

\item The assertion coverage by constrained random
    simulation is significantly improved by guiding the simulation through 
	formal methods proposed in Chapter~\ref{chap3}.

\item The scalability and efficiency of formal and
    semi-formal verification is significantly enhanced in terms of 
	expressibility and performance through the context-sensitive 
	specification style. 

\item The open specification style resulting as an 
	outcome of annotating temporal operators has significant space 
	advantages in model checking, since it allows validation of modules 
	in isolation under specific environment scenarios.

\end{itemize}

\noindent
Interestingly, the research presented in this thesis opens up a lot of 
future avenues. The concept of property-guided simulation presented in 
Chapter~\ref{chap3} may be extended to handle different types of 
specification paradigms, viz. history-based, state-based. We need to study 
the formal methods for this approach when we move to such paradigms, 
and their effects on the overall performance. One of the main bottlenecks 
of our approach of Chapter~\ref{chap3} is in realizability checking, 
and there are apprehensions of degraded performance if we need 
a realizability check at every iteration. There are two workarounds for 
this: (a) consider a small subset of properties when performing the 
realizability check, or (b) devise good approximate realizability 
checking algorithms that can serve our purpose. We are working on these 
to make our framework more scalable and efficient. 

\noindent
Context-sensitive specification styles have been quite well appreciated 
in the assertion IP design community, particularly because of its 
convenience in modeling complicated behavioral requirements. In 
addition, we expect that the improvements in performance demonstrated by us 
will also generate much interest. As we realized, this new style of having 
small context state machines and context-sensitive properties is still at 
a very nascent state of research and will open up lots of new research 
problems. One of the directions of our current research on 
context-sensitive specifications is to find out assumptions 
and methodologies that 
can make this new specification paradigm more amenable to FPV. In addition, we 
are analyzing the effects of context state machines in a Counterexample 
guided abstraction refinement framework (CEGAR), which is an 
established practice in the early stages of the validation flow today
(Figure~\ref{fig1.2}). We believe the context state machines can help 
achieve faster closure in the refinement/abstraction loop and in the 
verification of spurious counterexamples as well, since these are macro-models 
of the actual implementation. The full potential of this specification 
paradigm is still under exploration, and we believe that this new 
specification style, when imbibed into the mainstream formal specification 
flow, will produce significant benefits. 

\noindent
As we refine the design in steps by decomposing
its functionality into that of its blocks and adding level specific
details, we must also refine the formal specification -- starting from the
architectural specification, down to the block level specifications,
and finally down to the specifications of unit level modules. Current 
specification languages are not well suited for this decomposition, as 
mentioned in Chapter~\ref{chap7}, and we believe that our methodology 
of assume-guarantee specification, separating the assume from the 
guarantee is of significant value. We have analyzed the FPV effects of 
this new specification paradigm. As future work, we wish to take up 
two directions (a) consistency analysis of this specification paradigm, 
and (b) effects of this specification style in a semi-formal framework. 
While the methodology behind the first is still to be worked out, the 
second problem is intuitively more easy to approach. For this, we feel that 
we can translate the environment scenarios annotating the temporal operators 
into a simulation test plan to be used for DPV. This test plan will be able to 
capture the correct test-bench conditions for which a given property is to 
be validated, thereby increasing assertion coverage.

\noindent
Last but not the least, our efforts in Chapter~\ref{chap8} in 
devising a fully-functional DPV framework for UML Statechart validation 
is expected to open up a lot of future research avenues. Till date, most 
of the research in this direction has been based on support for feature-rich 
languages and the corresponding FPV methods to handle large and complex 
software state spaces. Our proposal in this direction is novel, in the sense, 
that it extends the current trends in hardware validation to a software 
architecture. There are lots of issues still to be solved, like 
multi-threading, concurrency etc. and hence, we believe, a bundle of new open 
research problems to emerge as we migrate our formal methods to software. 
As future work, we intend to investigate the effects of non-vacuous test 
case generation in the context of Action-LTL properties with data 
attributes and parameterized events to improve the overall assertion 
coverage of the DPV process.

\noindent
There has been several important realizations during the conceptualization of
this thesis. We mention some of these below:
\begin{enumerate}

\item Model checking has hit a complexity barrier which we cannot adequately
    overcome by adopting engineering ideas in FPV tools. We need new formal 
	methods for abstraction and pruning.

\item If we can develop large designs by decomposition, then we should also
    be able to verify large designs by decomposition. If design
    decomposition can be conceived by human beings, then specification
    decomposition can also be conceived by human beings. We only need to
    provide formal methods to verify the design decompositions 
	against their corresponding specifications under 
	the right environment conditions.

\item Verifying the consistency and completeness of formal property
    specifications will become an increasingly significant problem. We
    need good tools for this purpose, if property verification 
    is to be adopted by many. 

\item Specification guided automatic test generation and simulation will
    become a reality. We will have to look for bugs in the right places
    -- it will be increasingly infeasible for simulation to look for
    them everywhere.

\end{enumerate}
\noindent
All these point out to the fact that property verification technology is 
poised at an interesting state today. Many chip design companies realize 
that this technology holds tremendous potential, and yet are unable to 
seamlessly adopt the technology into their validation flow because of the 
missing pieces in the integrated validation flow. We believe this thesis 
has addressed some of these missing pieces in a truly effective way.
