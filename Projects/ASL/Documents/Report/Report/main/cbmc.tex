\subsection{CBMC}
 
The bounded model checking Implementations for C programs is CBMC developed 
at CMU. CBMC formally verifies ANSI-C programs and checked the properties of 
C including pointer safety, array bounds and user-provided assertions. As CBMC 
architecture is based on the BMC model,CBMC forms a transition relation for a
complex state  machine  with its specifications are unwound to reduce a boolean
formula which is then fed to a SAT solver. The SAT solver checks for the counter
example where the specifications contradicts with the boolean formula 
implementations. If no such counter example obtained within the data type range 
bound then the  verification  is successful otherwise a counterexample showing 
the trace of  states with violating specifications is depicted.
 
 
CBMC behaves like a compiler and takes .c files on the command line. Then like a 
linker merges the function definition from various .c files and performs symbolic 
simulation on the program.
Let us consider the following C program, named Test1.c
\begin{verbatim}
#include <stdio.h>
 
int main()
{
  int a[4] = {1, 2, 3, 4, 5};
  
   printf("%d", a[3]);
}
\end{verbatim}

Now we can run this program using CBMC in the following way.

cbmc Test1.c

CBMC transforms this design into CNF and passes it to a SAT solver.
If SAT solver detects that the design is not valid then there is a bug in the 
program. Then it prints a counterexample that shows a trace where the property 
is violated. In the above code, cbmc will show that verification is successful
and if it was not successful it would show us a counterexample.\\
 
Let us consider another example:--$>$
\begin{verbatim}
#include <stdio.h>
  
int main()
{
     int a, b;

     __CPROVER_assume(a <= 5);
     __CPROVER_assume(b <= 6);
     __CPROVER_assert(a+b <= 11,"");

     return 0;
}

\end{verbatim}

Cbmc exhaustively searches for counter example  and finally reports Verification 
Failure with this following counterexample:
\begin{verbatim}
Building error trace

Counterexample:

State 18 file Test3.c line 4 function main thread 0
----------------------------------------------------
  a=-2147483639 (10000000000000000000000000001001)

State 19 file Test3.c line 4 function main thread 0
----------------------------------------------------
  b=-2147483644 (10000000000000000000000000000100)

Violated property:
  file Test3.c line 8 function main
  assertion
  a + b <= 11
\end{verbatim}

As cbmc supports overflow in integer arithmetic, the sum of a and b are overflowed
and gives 13 that is greater than 11.


\textbf{Verifying Modules}
 
Generally programs starts with main function but in embedded software, there are
different entry points in the program.. Consider the following code
\begin{verbatim}
#include <stdio.h>
 
int main()
 {   
 }
 
int product()
 {
 }
\end{verbatim}
 
Here is a main functions and one sub function product. We want to set entry point 
to the product function, then we can write

cbmc Test2.c --function product
 
\textbf{Loop Unwinding}
 
CBMC unwinds the loop. As CBMC follows Bounded Model Checking architecture, all
loops are unwinds upto certain depth, within which it guarantees that all bugs 
are found. Consider the following code, Prime.c :
 
\begin{verbatim}
 int prime()
 {
    int i = 2;
   while(1)
   {
     checkPrime(i);
       i = i + 1;
   }
 }
\end{verbatim}
CBMC cannot stop unwinding this design. The built-in simplifier cannot determine
the upper bound to stop. But the unwinding bound can be given as the command line
argument:

cbmc prime.c --function prime --unwind 8 --decide

\textbf{Basic data types}
 
CBMC supports the scalar data types including Bool. CBMC allows int 32 bits
wide, short int 16 bit wides and char is 8 bits wide. Char is by default 
signed but using a  command line options char an be changed  to unsigned. 
CBMC also supports floating point data types like float, double  and long double.
CBMC uses fixed point representation for floating point type.  By default float 
 is 32 bits wide, double and long double are 64 bits wide. \\
 
\textbf{ Operators:}  \\
 
\textbf{Bololean operators} \\
 
 Cbmc supports and, or,  not boolean operators. \\
 
\begin{center}

\begin{tabular}{|| c | c ||}

\hline

OPERATOR & DESCRIPTION  \\ [0.5ex]
\hline
\hline
!a & negation \\
\hline
a $ \&\& $ b & and \\
\hline
a $||$ b & or \\
\hline

\end{tabular}

\end{center}
\textbf{Integer Arithmetic Operators} \\
 
CBMC supports all integer Arithmetic operators. Cbmc also checks Arithmetic 
overflow  in case of signed operands and division by zero for division and 
remainder operators. \\
begin{center}

\begin{center}
\begin{tabular}{|| c | c ||}

\hline

OPERATOR & DESCRIPTION  \\ [0.5ex]
\hline
\hline
-a & negation \\
\hline
a $ + $ b & sum \\
\hline
a $ - $ b & subtraction \\
\hline
a $ * $ b & multiplication \\
\hline
a $ / $ b & division \\
\hline
a $ \% $ b & remainder \\
\hline
a $ << $ b & bit-wise-left-shift \\
\hline
a $ >> $ b & bit-wise-right-shift \\
\hline
a \&  b & bit-wise-and \\
\hline
a $ | $ b & bit-wise-or \\
\hline
a $ ^ $ b & bit-wise-xor \\
\hline
a $ < $ b & relation \\
\hline
a $ <= $ b & relation \\
\hline
a $ > $ b & relation \\
\hline
a $ >= $ b & relation \\
\hline
a $||$ b & or \\
\hline

\end{tabular}

\end{center}

\textbf{Floating Point Arithmetic operators} 
 
Cbmc supports addition, subtraction, multiplication and relational operations
on float, double and long double variables. CBMC does not support overflow in
case of floating point arithmetic. The example stated above where failure occur
due to arithmetic overflow will be successfully verified by CBMC for floating 
point arithmetic. \\
 
\begin{verbatim}
#include <stdio.h>

int main()
{
     float a, b;

     __CPROVER_assume(a <= 5.0);
     __CPROVER_assume(b <= 6.0);
     __CPROVER_assert(a+b <= 11.0,"");

     return 0;
}
\end{verbatim}
 
\textbf{Type casts}
 
CBMC supports Arithmetic type casts. For unsigned data types overflow is not 
checked but for signed data types the overflow condition is checked. \\

\textbf{Control Flow Statement}
 
CBMC supports the use of all conditional Statement as supported by ANSI-C standard.
CBMC checks that a function having non-void return type must return a value 
according to the return type . \\
 
\textbf{NON-Determinism}
 
CBMC supports non-deterministic choice of functions and is defined them with 
the prefix nondet. For example,
\begin{verbatim}
 int nondet_int();
\end{verbatim}

This statements enable cbmc to return any non-deterministically taken int value.
CBMC exhaustively checks all int value from this nondeterministic set. \\


\textbf{Assumptions and Assertions}

CBMC assumptions restricts the program traces that follows the assumptions.
It takes boolean expression, Consider the following example
 
\begin{verbatim}
 int eval = nondet_int();
  
 __CPROVER_assume(eval > 1 && eval <= 100);
\end{verbatim}
In the course of execution only the value of eval will be between 1 to 100.
 
The assert statement also takes boolean Statement and CBMC checks whether this
condition is true for each run of the designed code. If the condition is true 
for each runs then only the assertions can pass. Consider the following example:
 
\begin{verbatim}
 int eval = nondet_int();
  
 __CPROVER_assume(eval == 110);
 assert(eval > 50);
\end{verbatim}
 
The above assertion will pass but the following example

\begin{verbatim}
 int eval = nondet_int();
  
 __CPROVER_assume(eval == 10);
 assert(eval > 50);
\end{verbatim}
 
Will fail as there is no such path that satisfies this property.

When using the assume statement, it is ensured that at least one program trace 
is supported by this assume conditions but when we use assert we want to establish
that the property is true for each program trace. \\
 
\textbf{Arrays}
 
CBMC supports Arrays as defined by the ANSI-C standard including multidimensional
and dynamically allocated array. CBMC checks for upper bound and lower bound of
the array to warn the occurrence of overflow and underflow condition.
 
Let us consider the following example:--$>$
\begin{verbatim}
#include <stdio.h>

int main()
{
     int a[10] = {5};

    printf("%d",a[20]);

     return 0;

}
\end{verbatim}
To check the array bound we have to run this code with the following option

cbmc Test.c --bounds-check

The verification fail showing the following counterexample:--$>$
\begin{verbatim}
Building error trace

Counterexample:

State 18 file Test1.c line 5 function main thread 0
----------------------------------------------------
  a={ 0, 0, 0, 0, 0, 0, 0, 0, 0, 0 } ({ 00000000000000000000000000000000, 
  00000000000000000000000000000000, 00000000000000000000000000000000, 
  00000000000000000000000000000000, 00000000000000000000000000000000, 
  00000000000000000000000000000000, 00000000000000000000000000000000, 
  00000000000000000000000000000000, 00000000000000000000000000000000, 
  00000000000000000000000000000000 })

State 19 file Test1.c line 5 function main thread 0
----------------------------------------------------
  a={ 5, 0, 0, 0, 0, 0, 0, 0, 0, 0 } ({ 00000000000000000000000000000101, 
  00000000000000000000000000000000, 00000000000000000000000000000000, 
  00000000000000000000000000000000, 00000000000000000000000000000000, 
  00000000000000000000000000000000, 00000000000000000000000000000000, 
  00000000000000000000000000000000, 00000000000000000000000000000000, 
  00000000000000000000000000000000 })

Violated property:
  file Test1.c line 7 function main
  array `a' upper bound in a[(signed long int)20]
  FALSE

VERIFICATION FAILED

\end{verbatim}

 
Apart from this cases CBMC supports many more features as shown in the following table.
 
\begin{center}

\begin{tabular}{|| c | c ||}

\hline

OPTION & DESCRIPTION  \\ [0.5ex]

\hline
\hline
$\textendash$ $\textendash$show$\textendash$parse$\textendash$tree     &       show parse tree \\
\hline
$\textendash$ $\textendash$show$\textendash$symbol$\textendash$table    &      show symbol table \\
\hline
$\textendash$ $\textendash$show$\textendash$goto$\textendash$functions   &     show goto program \\
\hline
$\textendash$ $\textendash$bounds$\textendash$check    &           enable array bounds checks \\
\hline
$\textendash$ $\textendash$div$\textendash$by$\textendash$zero$\textendash$check     &     enable division by zero checks \\
\hline
$\textendash$ $\textendash$pointer$\textendash$check         &     enable pointer checks \\  
\hline
$\textendash$ $\textendash$memory$\textendash$leak$\textendash$check      &    enable memory leak checks \\
\hline\hline
$\textendash$ $\textendash$signed$\textendash$overflow$\textendash$check   &   enable arithmetic over$\textendash$ and underflow checks \\
\hline
$\textendash$ $\textendash$unsigned$\textendash$overflow$\textendash$check  &  enable arithmetic over$\textendash$ and underflow checks \\
\hline
$\textendash$ $\textendash$float$\textendash$overflow$\textendash$check      & check floating-point for +/-Inf \\
\hline
$\textendash$ $\textendash$nan$\textendash$check               &   check floating$\textendash$point for NaN \\
\hline
$\textendash$ $\textendash$no$\textendash$assertions           &   ignore user assertions \\
\hline
$\textendash$ $\textendash$no$\textendash$assumptions          &   ignore user assumptions \\
\hline
$\textendash$ $\textendash$error$\textendash$label label       &   check that label is unreachable \\
\hline
$\textendash$ $\textendash$cover CC                &   create test-suite with coverage criterion CC \\
\hline
$\textendash$ $\textendash$mm MM                   &   memory consistency model for concurrent programs \\
\hline
$\textendash$ $\textendash$function name         &     set main function name \\
\hline
$\textendash$ $\textendash$property id           &     only check one specific property \\
\hline
$\textendash$ $\textendash$program$\textendash$only          &     only show program expression \\
\hline
$\textendash$ $\textendash$show$\textendash$loops            &     show the loops in the program \\
\hline
$\textendash$ $\textendash$depth nr              &     limit search depth \\
\hline
$\textendash$ $\textendash$unwind nr             &     unwind nr times \\
\hline
$\textendash$ $\textendash$unwindset L:B,...     &     unwind loop L with a bound of B
                                                       (use --show-loops to get the loop IDs) \\
\hline                        
$\textendash$ $\textendash$show$\textendash$vcc               &    show the verification conditions \\
\hline
$\textendash$ $\textendash$slice$\textendash$formula         &     remove assignments unrelated to property \\
\hline
$\textendash$ $\textendash$unwinding$\textendash$assertions   &    generate unwinding assertions \\
\hline
$\textendash$ $\textendash$partial$\textendash$loops          &    permit paths with partial loops \\

\hline

\end{tabular}

\end{center}
 