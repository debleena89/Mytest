\section{Methods for Software Verification}

\subsection{Theorem Proving}

When we want to write a program we want to get desired
output from that program. Or when we have a program we want to know what will be 
the output of this program. Taking some considered value we go through the code
and come to some conclusion. Now how we can we understand that our inferred result
is correct or not. Or for any possible inputs, the outputs can be predicted.
Suppose If I have a code to make square of some Integer non negative number ,
we can say that for any non-negative integer value the output will be the 
square of the input number. So there are techniques called theorem prover
that says how to prove a computer program.


By using Theorem prover we want to have 
a mathematical proof of  an expression. This expression is called 
conjecture and stated in formal language. To prove such an 
expression we need to have Axioms and Inference rules. Axioms are
conjecture which are considered to be true. Inference rules says that if I have
fact1, fact2..factn to be true then we can infer New fact from them.

For example

\begin{verbatim}
 a * (b * c) = (a * b) * c : Axioms
 
 a * e = b * e : Inference Rules
 
 then we can infer that  a = b.  : New Fact
\end{verbatim}

theorem prover works in the following way:

a) we have a sets of axioms A

b) Given a conjecture P.

c) To prove whether A $\models$ P ,i.e. prove A $\cup$ ${\sim P}$ unsatisfiable

A and P are written in formal language that is First Order language. Now the 
derived formula is deduced by Inferring from the sequence of axioms. 
As Axioms are true and valid and if the expression to prove can be 
derived by inferring from the sequence of axioms is also valid.

   we can precisely formulate this as HOARE TRIPLES where we have
\begin{equation}
 \{S_o\} P \{S_final\}
\end{equation}

          
  Where $S_0$ is the sets of initial condition \\
        P can be an expression , function call or any block of code \\
        $S_final$ gives the sets of the final result \\
        
In Hoar's calculus Axioms and Inference rules are used to derive $S_final$ based
on the $S_0$ and P. $S_0$ and $S_final$ basically validates the program. If only P gets
changed $S_final$ has to change but $S_0$ can remain unchanged. For different $S_0$ 
initial condition for a particular P we will get different $S_final$. If $S_0$ kept
same and P gets changed , we will get different $S_final$.

\textbf{Application}

Theorem prover was developed to help the Mathematicians. It can search equivalent
theorem from a database of mathematical theorems. Moder application of theorem 
prover can be in the field of hardware and software verification. Automated theorem
provers can be applied in other field also like network security.
          
         

\subsection{Software Model Checking}

Software Model Checking is the algorithmic analysis of a program to determine
if the statement of the program are correctly designed to give the expected 
result. A computer program consists of some finite number of lines of code.
Say if a program has n number of lines of code then it requires n number of
program counter locations. Program counter stores the address of the instructions
to be executed next. Valuation of the variables is called the states of the program.
Now a model of a program  consists of the control flow of the program through the 
combination of states and program counter. 
Configuration transition graph describes how the state changes in course of the
execution of the program. Each node gives a combination of state and program 
counter. The edge depicts the movement from one state to another. Configuration
graph must show every state of the program. The model checking algorithms 
determines exhaustively the reachability of the states. If the graph has a finite
number of configuration then it says that the program will terminate. consider 
the following code :

\begin{verbatim}

1: x := 3;
  
2:  while(x >= 0)
   
3:     if(x >= 1) then 4: x := -5;
     
                else 5: x := 2;
6:          
     
\end{verbatim}

The configuration transition graph is depicted below

\begin{verbatim}

   <s,1> --> <x = 3, 2> --> <x = 3, 3> --> <x = 3, 4>
   --> <x = -5, 5> --> <X = -5, 6>
   
\end{verbatim}


If the configuration graph reaches such a state  where there is no instructions
to execute then the program will terminate. If such a state is not reachable 
then there will be infinite state transition.

  Model checking tool checks for such reachable state. If the tool checks that 
  this snippet of code is unable to produce expected result then it gives a 
  counter example to illustrate a situation when such circumstance can occur.
  model checking tool depicts an execution trace which says for which state 
  configuration the program cannot produce expected result.
  
\subsection{Bounded Model Checking}
  
\textbf{Background:}
Bounded Model checking was first proposed by Biere et al in 1994. It follows
the principle of model checking but upto a depth say k. So it is called bounded
model checking. It explores the reachable states within that depth k. If a bug is
found within K depth BMC gives a counterexample of the trace of the states. BMC
cannot report if a bug could be present in greater than K depth. This depth value
is called Completeness threshold beyond which BMC is unable to check.

\textbf{Workings}
BMC unwinds the path k times and formulate Linear Temporal Logic for each path. 
It then checks whether there exists such a trace that contradicts them, gives 
the counterexample. Such a trace is called witness for the property.

We are considering here two path quantifiers E and A \\
E denotes the expectation of correctness of the LTL formula over all the paths \\
And \\
A denotes the expectation of correctness of the LTL formula over some paths.

M $\models$ Af means , M satisfies f  for all of the paths and M $\models$ Ef means there exist
at least one path that satisfies f. The formula are presented in negated normal form(NNF).

The basic idea of bounded model checking is to consider only a finite length of 
the path, restricted to some bound k. Though the length of the path is finite, 
it can represent an infinite path, if the back loop is present in the path from 
the last state to any of its previous state. If  there is no such loop, then
the path is finite within bound k.


We call a path $\pi$, a k-loop if there exists  k $>=$ l $>=$ 0 , for which $\pi$ is a 
(k,l) loop. Here, k loop signifies that, the semantics of model checking is 
defined under bounded traces. In the bounded semantics, we consider a finite
length of a path. Precisely we only explore first k$+$1 states ($S_0$,..$S_k$)of a path
to establish the validity of the LTL formula along the path.

\textbf{Bounded Semantics for a loop}

Let k $>$ 0 and $\pi$ be a k-loop. Then  an LTL formula f is valid along the path 
$\pi$ with bound k (in symbol $\pi$ $\models$ $\pi$k) iff $\pi$ $\models$ f.

Now let us consider $\pi$ is not a k-loop. Then the formula f := $F_p$ is valid 
along the path $\pi$ in the unbounded semantics if we can find an index i $>$ 0, for
which p is not valid along the suffix $\pi$i of $\pi$.

\textbf{Bounded Semantics without a loop}

Let k $>$ 0, and let $\pi$ be  path that is not a k-loop. Then an LTL formula f 
is valid along $\pi$ with bound k.


\textbf{Reducing bounded model checking to SAT}

In this section we will discuss how to reduce bounded model checking to 
propositional satisfiability. Precisely this technique illustrate how propositional
SAT solver can be used efficiently to perform bounded model checking.

Let us consider, a kripke structure M, an LTL formula f and a bound k and we want 
to construct a propositional formula $[[ M,f ]]_k$.

$S_0$,...$S_k$ be a finite sequence of states on a path $\Pi$ .

Each $S_i$ represents a state at time stamp i and consists of an assignment of 
truth values from the set of state variables. The formula $[[M,f]]_k$ can be defined as
the 3 separate formula.

The first component defines a propositional formula $[[M]]_k$, that contains
$S_0$...$S_k$ states to form a vital path starting from an initial state.

For a Kripke structure M, k $>=$ 0.
\begin{equation}
   [[M]]_{k} := I(s_0) \wedge \bigwedge_{i=0}^{k-1} T(s_i,s_i+1)
\end{equation}   
  The shape of the path $\Pi$ plays a vital role as the translation of an LTL 
  formula depends on the shape. The propositional formula is considered to be 
  true if and only if a transition occurs from state $s_k$ to state $S_1$.
  
  the second component is a loop condition, that is propositional formula evaluates 
  to true only if the path $\pi$ contains a loop.
  
  
  The \textbf{Loop Condition} $L_k$ is true if and only if there exists a back loop
  from state $S_k$ to a previous state or to itself.
  
\begin{equation}
     L_k := \bigvee_{l=0}^{k} _{l}L_{k}
\end{equation}
   
           
          
  The third component defines a propositional formula that constraints $\pi$ to
  satisfy P.
