\chapter*{}
\vspace*{-2in}
%Abstract\markboth{Abstract}{Abstract}}
\begin{center}
\Large \bf {Abstract}
\end{center}
%\vspace{-.2in}

\noindent
In recent times, most leading
chip design companies are seriously investigating the possibility of
integrating property verification into their pre-silicon validation flows.
Property verification allows the designer to express the key
correctness requirements of a design in terms of formal properties and
verify them over a given implementation. 
Property verification is predominantly used in two forms in pre-silicon
validation, namely (a) dynamic property verification (DPV), and
(b) static Formal Property Verification (FPV).

\noindent
Existing property verification technology is poised at an interesting point. 
The benefits of property verification have been established quite
emphatically in the last decade. Researchers have analyzed several
historically significant failures and have shown that the use of property
verification could have detected the bug in the design.
Yet, the adoption of property verification techniques into the 
pre-silicon validation flow of chip design companies has been retarded 
by the following factors:
\begin{itemize}

\item {\em FPV Capacity:} Current FPV techniques do not scale
    beyond small circuit modules. 

\item {\em DPV Coverage:} The main criticism of the DPV approach is that 
	only those behaviors that are covered by simulation are examined for 
	property violation. 

\end{itemize}

\noindent
This thesis has two main motivations:

\begin{itemize}

\item We believe that the scalability and efficiency of formal and 
	semi-formal verification can be improved by adopting specification styles 
	that syntactically facilitate abstractions and pruning.

\item We believe that the assertion coverage in DPV 
	can be improved by guiding the simulation through formal 
	methods.
\end{itemize}
\noindent
The main objective of this thesis is to study the above issues
and propose methods for accelerating formal, semi-formal and dynamic 
property verification.
In particular, we have the following contributions:

\begin{itemize}
\item {\em Accelerating Property Coverage in DPV:} We have defined an 
	automated methodology that can analyze formal properties and produce 
	tests that trigger them. This has been integrated within a 
	constrained random test architecture to accelerate coverage of 
	corner case assertions during DPV.

\item {\em Context-sensitive specifications:} A significant fraction
    of the correctness requirements in standard protocol descriptions 
	are {\em context-sensitive}, i.e. they apply only when the 
	protocol is in specific contexts. We have formalized a modeling 
	style for expressing such properties and proposed algorithms 
	for formal and semi-formal verification 
	and consistency analysis for such properties.

\item {\em Annotating Temporal Operators for Modular FPV:}
	To ameliorate the state explosion
    problem, there has been a paradigm shift from a system level FPV
    perspective to a modular one. To capture module-level specifications, 
	we have proposed an extension of Linear Temporal Logic by annotating 
	temporal operators with input constraints. In addition, 
	we have proposed a symbolic BDD-based algorithm for verifying such 
	properties. 

\item {\em An integrated DPV platform for UML Statechart validation:} 
	We have developed a complete DPV platform for verifying temporal 
	requirements over UML Statecharts. The DPV framework allows the user 
	to specify correctness requirements in a rich assertion specification 
	language and verifies them during simulation. 

\end{itemize}
\noindent
We believe that the formal methods presented in
this thesis will lead to wider adoption of property verification techniques 
in the design validation flow.
